\documentclass[11pt]{article}
\usepackage[utf8]{inputenc}
\usepackage[margin=1in]{geometry}
\usepackage{amsmath, amssymb, amsfonts}
\usepackage{enumitem}
\usepackage{tcolorbox}

\title{CSE400: Fundamentals of Probability in Computing \\ \large Lecture 6: Discrete RVs, Expectation and Problem Solving}
\author{Dhaval Patel, PhD}
\date{January 22, 2025}

\begin{document}

\maketitle

\section{Introduction to Random Variables (RVs)}
[cite_start]A \textbf{random variable X} on a sample space $\Omega$ is a function $X: \Omega \to \mathbb{R}$ that assigns a real number $X(\omega)$ to each sample point $\omega \in \Omega$[cite: 48].
\begin{itemize}
    [cite_start]\item \textbf{Discrete RVs:} Take values in a range that is finite or countably infinite[cite: 49].
    [cite_start]\item \textbf{Distribution Visualization:} Can be represented as a bar diagram where the x-axis shows values and the height shows $Pr[X=a]$[cite: 104, 105, 106].
\end{itemize}

\section{Probability Mass Function (PMF)}
[cite_start]A random variable is discrete if it can take on at most a countable number of values[cite: 204].
\begin{itemize}
    [cite_start]\item \textbf{Definition:} The function $P_X(x_k) = P(X = x_k)$ for $k = 1, 2, 3, \dots$ is the PMF of X[cite: 221, 222].
    [cite_start]\item \textbf{Property:} The sum of all probabilities must equal 1: $\sum_{k=1}^{\infty} P_X(x_k) = 1$[cite: 233].
\end{itemize}

\section{Bayes' Theorem Recap}
\begin{tcolorbox}
\[ Pr(B_i|A) = \frac{Pr(A|B_i)Pr(B_i)}{\sum_{i=1}^n Pr(A|B_i)Pr(B_i)} \]
\end{tcolorbox}
\begin{itemize}
    [cite_start]\item \textbf{A priori probability:} $Pr(B_i)$ - formed from presupposed models[cite: 291].
    [cite_start]\item \textbf{Posteriori probability:} $Pr(B_i|A)$ - calculated after observing event A[cite: 292].
\end{itemize}

\section{Independence}
\begin{itemize}
    [cite_start]\item \textbf{Two Events:} A and B are independent if $Pr(A|B) = Pr(A)$ and $Pr(B|A) = Pr(B)$, implying $Pr(A,B) = Pr(A)Pr(B)$[cite: 448].
    [cite_start]\item \textbf{Three Events:} A, B, and C are mutually independent if[cite: 460]:
    \begin{itemize}
        [cite_start]\item $Pr(A,B) = Pr(A)Pr(B)$, $Pr(B,C) = Pr(B)Pr(C)$, $Pr(A,C) = Pr(A)Pr(C)$[cite: 462, 463].
        [cite_start]\item $Pr(A,B,C) = Pr(A)Pr(B)Pr(C)$[cite: 463].
    \end{itemize}
\end{itemize}

\section{Types of Discrete Random Variables}

\subsection{Bernoulli Random Variable}
[cite_start]Models an experiment with only two outcomes: Success (1) or Failure (0)[cite: 535, 544].
\begin{itemize}
    [cite_start]\item \textbf{PMF:} $P_X(1) = p$ and $P_X(0) = 1-p$, where $p \in (0,1)$[cite: 556, 557].
    [cite_start]\item \textbf{Examples:} Single coin toss, email spam classification[cite: 572, 574].
\end{itemize}

\subsection{Binomial Random Variable $B(n,p)$}
[cite_start]Models the number of successes in $n$ independent trials, each with success probability $p$[cite: 597, 599].
\begin{itemize}
    [cite_start]\item \textbf{PMF:} $p(i) = \binom{n}{i} p^i (1-p)^{n-i}$ for $i = 0, 1, \dots, n$[cite: 612].
    [cite_start]\item \textbf{Examples:} Correct answers on a test, number of defective items in a sample[cite: 626, 627].
\end{itemize}

\subsection{Geometric Random Variable}
[cite_start]Models the number of independent trials required until the first success occurs[cite: 644, 645].
\begin{itemize}
    [cite_start]\item \textbf{PMF:} $P_X(X=n) = (1-p)^{n-1} \times p$ for $n=1, 2, \dots$[cite: 659].
    [cite_start]\item \textbf{Examples:} Tosses until first head, attempts until first sale[cite: 669, 670].
\end{itemize}

\subsection{Poisson Random Variable}
[cite_start]A discrete RV with parameter $\lambda > 0$[cite: 695].
\begin{itemize}
    [cite_start]\item \textbf{PMF:} $p(i) = P\{X=i\} = e^{-\lambda} \frac{\lambda^i}{i!}$ for $i = 0, 1, 2, \dots$[cite: 702, 703].
    [cite_start]\item \textbf{Approximation:} Used for $B(n,p)$ when $n$ is large and $p$ is small such that $n \times p$ is moderate[cite: 725, 726].
    [cite_start]\item \textbf{Examples:} Misprints on a page, customers entering a post office daily[cite: 740, 744].
\end{itemize}

\end{document}